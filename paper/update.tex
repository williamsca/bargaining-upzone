\documentclass[12pt]{article}
\usepackage[right=1.25in,left=1.25in,top=1.1in,bottom=1.1in]{geometry}
\usepackage{lscape}
\usepackage{hyperref}
\hypersetup{colorlinks, citecolor=blue, filecolor=blue, linkcolor=blue, urlcolor=blue}
\usepackage{graphicx}
\usepackage{url}
\usepackage[round]{natbib}
\usepackage{amsmath,amsthm}
\usepackage{float}
\usepackage{natbib}
\usepackage{booktabs}
\usepackage[skip=5pt plus1pt, indent=20pt]{parskip} % USE THIS TO SET SPACING
\usepackage{amssymb} %% Necessary, just for the checkmark command  in tables.
% \usepackage{setspace}
% \onehalfspacing
\title{ \vspace*{-2.5cm} \hspace*{-0.5cm} }
\author{Colin Williams\thanks{University of Virginia.
\href{mailto:chv7bg@virginia.edu}{chv7bg@virginia.edu}}}
\date{ \vspace*{0.5cm} \today} % \textbf{Preliminary and Incomplete. \ Please do not cite or circulate.}

%%%%%%%%%%%%%%%%%%%%%%%%%%%%%%%%%%%%%%%%%%%%%%%%%%%%%%%%%%%%%

\begin{document}

\bgroup
\let\footnoterule\relax

%\begin{singlespace}
\maketitle

\begin{abstract}
\noindent 
\end{abstract}
%\end{singlespace}

% These commands clear the title page
% \thispagestyle{empty}
% \clearpage
% \egroup
% \setcounter{page}{1}

\noindent 

\section{Proffer Reform}  \label{sec:data}

Last week, I presented evidence that housing prices fell in Virginia after the Proffer Reform Act of 2016. 

This was surprising. In the simplest urban models, housing prices are pinned down by a combination of a spatial indifference condition and an ``agricultural hinterland'' with perfectly elastic supply. Any change in the cost of building a house will be capitalized into land prices without changing housing prices. The intuition is that, modulo commuting costs, housing is perfectly substitutable across space, so demand at any location is perfectly elastic.

Evidently, these simple models are not a good description of counties in Virginia. Changes in the price of an input to construction -- the right to build -- appear to be passed through to final prices. This finding is consistent with downward-sloping housing demand at the county level.

\subsection{Proffers Fell After the Reform}

The Proffer Reform Act imposed constitutional limitations on the ability of jurisdictions to collect cash proffers. Although many jurisdictions put a temporary pause on rezonings in order to determine how the law would affect their proffer systems, it appears that they ultimately reduced proffer amounts substantially. Figure \ref{fig:plot_proffers} shows per-unit average proffers for residential rezonings in three primarily suburban counties (Hanover, Goochland, and Chesterfield) between 2014 and 2020.

\begin{figure}[h]
    \caption{Per-Unit Average Proffers}
    \includegraphics[width=\textwidth]{figures/plot_proffers.png}
    \label{fig:plot_proffers}
\end{figure}

There is a clear decline in approved units between mid-2016 and 2018 as counties, developers, and courts interpreted the new law. However, by mid-2018, these counties approved substantial developments with zero cash proffers, a practice that rarely occured prior to 2016.

\subsection{Treatment Effect Varies with Pre-Reform Reliance on Proffers}
To investigate the effect of the Proffer Reform on prices, I defined three groups of Virginia counties based on their share of local revenue derived from proffers in 2016. Those with a proffer share of revenues above the state average (0.33\%) were ``High Proffer''; those with a positive but below-average share were ``Low Proffer''; and those with zero proffer revenue were ``No Proffer''. I also restrict the sample to counties for which I observe the Zillow House Value Index continuously between 2010 and 2021.\footnote{The Synthetic Difference-in-Differences method requires a balanced panel.} I then ran three separate synthetic diff-in-diff regressions, constructing the synthetic control from the balance of continental US counties in each case. I followed this procedure because I was not sure how to think about spillovers across counties due to changes in proffer policy.

I report summary statistics below:

% latex table generated in R 4.3.1 by xtable 1.8-4 package
% Mon Oct 23 16:47:06 2023
\begin{table}[ht]
\centering
\begin{tabular}{llll}
  \hline
Variable & High Proffer & Low Proffer & No Proffer \\ 
  \hline
Number of Counties & 16 & 7 & 16 \\ 
  Population & 193,666 & 102,335 & 96,270 \\ 
   & (264,279) & (149,563) & (101,203) \\ 
  Zillow HVI (\$) & 310,854 & 266,700 & 235,106 \\ 
   & (88,505) & (113,978) & (168,909) \\ 
  HPI (2000 = 100) & 412 & 421 & 471 \\ 
   & (171) & (219) & (318) \\ 
  Local Revenue (\$/capita) & 2,322 & 2,386 & 2,557 \\ 
   & (711) & (1,258) & (1,373) \\ 
  Proffer Revenue (\$/capita) & 24 & 2 & 0 \\ 
   & (30) & (3) & (0) \\ 
  Single-Family Permits Per Thousand (2010-2016) & 6 & 3 & 2 \\ 
   & (3) & (2) & (1) \\ 
  Multi-Family Permits Per Thousand (2010-2016) & 1 & 1 & 1 \\ 
   & (1) & (2) & (2) \\ 
   \hline
\end{tabular}
\caption{Summary Statistics} 
\end{table}


In general, high proffer counties tend to have higher average housing prices, larger populations, and to issue more single-family building permits. However, housing price growth is similar across the three groups.

Geographically, high proffer counties tend to locate in the inner suburbs of DC, though there is a fair amount of dispersion.

\begin{figure}
    \centering
    \caption{The Geography of Proffer Revenues}
    \includegraphics[width=\textwidth]{figures/map-proffer-groups.png}
\end{figure}

\section{Misc/Next Steps}

\flushleft\textit{Industry Background.} I spoke with Mike Vanderpool, a lawyer in NOVA, to get more context on the reform and its impacts. He kindly offered to read a draft of the eventual paper.

\flushleft\textit{Collecting Data.} Manually searching government minutes for details about proffers is tedious. To scale up, I've implemented a script that reads PDFs into a string of text and then then passes each page to the OpenAI API. I use the GPT 3.5 model to to identify relevant passages and summarize key information. (``The following text comes from a local government minutes. If it describes a rezoning process...'') I'm still assessing how accurate this method is. I could use an RA to check the results and track down raw data on county webpages.

\flushleft\textit{Unresolved Issues.} 
\begin{itemize}
    \item Are there differential impacts of reduced proffers on new and used housing? Even if the two are perfect substitutes, impacts could vary if proffers are targetted towards infrastructure/amenities that benefit primarily new housing. I believe I can answer this question using CoreLogic data. I have access -- thanks to the other W. Ben -- and plan to clean it up this week.
    \item What is the fiscal response? Do counties raise property taxes or cut spending?
    \item The High/Low/No Proffer grouping conflates a policy choice (the proffer amount and rezoning decision) with an exogenous demand curve for housing. It would be better to group counties by their average per-unit proffer rate.
\end{itemize}

\end{document}